In the context of this paper, `spherical packing' refers to the problem of packing as many spherical caps of a certain size onto the surface of a hyper-sphere of given dimension, subject to some constraint on intersection of caps.The spherical packing approach is a well-known method with applications to source encoding \cite{Shannon1959}. A more general investigation of the spherical packing problem without consideration for practical applications is discussed in \cite{Rankin1955}.

Essentially this problem can be distilled to the following questions. What is the probability that a collection of $l$ vectors distributed randomly over the surface of a sphere exist with a certain degree of orthogonality? What is this probability if we consider a larger number of $n>l$ vectors?

In this paper we discuss the application of a spherical packing method to the problem of selecting semi-orthogonal groups of MU-MIMO wireless users based on the wireless channel vectors that belong to those users as in \cite{Swannack2005}. The objective of the selection process is to group users, who receive transmissions concurrently from a common transmitter, such that interference between users is minimized. Interference reduction can be described in terms of orthogonality between wireless channel vectors. In this context, motivation for the spherical packing approach can be described intuitively. Channel vectors are associated to the caps that live on the surface of the sphere. The degree of orthogonality between channels is directly related to the size of the caps assumed, and the constraints placed on intersection of these caps. Assuming we have two caps on the spherical surface that are associated with channel vectors respectively. If the caps are subjected to the constraint that they do not intersect each other, then the larger the caps are, the more orthogonal the channel vectors are guaranteed to be. In this way, we have a method of relating orthogonality between user channel vectors and the geometric spherical packing model.

