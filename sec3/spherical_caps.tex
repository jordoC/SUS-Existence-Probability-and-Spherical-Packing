Orthogonality between pairs of vectors can be interpreted geometrically on the spherical surface in terms of spherical caps. A spherical cap is formed on the surface of the hyper-sphere of radius $\rho^+$ by by first extending the channel vector $\underline{h_i}$ onto the spherical surface as described in Section \ref{sec:chan_norm}. Then a cone, with its apex at $\mathcal{O}$, with half angle $\theta$, centered along the axis of $\underline{h_i}$ is intersected with the spherical surface. This intersection defines a spherical cap. $\mathcal{C}_i$. This process is illustrated in Fig. \ref{fig:spherical_cap}. 

considering a spherical surface with two vectors touching the surface of the sphere at arbitrary points on the sphere. A measure of orthogonality between these vectors can easily be calculated by forming the inner product between these vectors. Alternatively we could form a cone for each vector. The apex of each cone is the origin of the sphere, $\mathcal{O}$. The half angle of each cone is given by $\theta$.

\begin{figure}
\centering
    \begin{tikzpicture}%[font = \sansmath]
        %define origin, 1-height of cone, radius
        \coordinate (O) at (0,0);
        \def\r{2}
        \def\H{.6}
        
        \begin{scope}
        % ball background color
        \shade[ball color = blue, opacity = 0.5] (0,0) circle [radius = \r];
        
        % cone
        \begin{scope}
            \def\rx{0.71}% horizontal radius of the ellipse
            \def\ry{0.15}% vertical radius of the ellipse
            \def\z{0.725}% distance from center of ellipse to origin
        
            \path [name path = ellipse]    (0,\z) ellipse ({\rx} and {\ry});
            \path [name path = horizontal] (-\rx,\z-\ry*\ry/\z)
                                        -- (\rx,\z-\ry*\ry/\z);
            \path [name intersections = {of = ellipse and horizontal}];
        
            %% radius to base of cone in ball
            %\draw[fill = gray!50, gray!50] (intersection-1) -- (0,0)
            %  -- (intersection-2) -- cycle;
            %% base of cone in ball
            %\draw[fill = gray!30, densely dashed] (0,\z) ellipse ({\rx} and %{\ry});
        \end{scope}
        
        % label of cone
        %\draw (0.25,0.4) -- (0.9,0.1) node at (1.05,0.0) {$q$};
        
        % ball
        \draw (O) circle [radius=2cm];
        
        % label of ball center point
        \filldraw (O) circle (1pt) node[below] {$\mathcal{O}$};
        
        % radius
        \draw[densely dashed] (O) to [edge label = $\rho^+$] (-1.4,1.32);
        \draw[densely dashed] (O) -- (1.4,1.32);
        %label colatitude angle
        \draw[densely dashed] (0.5,0.5) arc [start angle = 45, end angle = 90, x radius = 7mm, y radius = 7mm];
        \draw (2.2, 0.8) -- (0.4, 0.6) node at (2.35,0.8) {$\theta$};
        
        % cut of ball surface
        \draw[red] (-1.35,1.47) arc [start angle = 140, end angle = 40, x radius = 17.6mm, y radius = 14.75mm];
        %\draw[red, densely dashed] (-1.36,1.46) arc [start angle = 170, end angle = 10,
        x radius = 13.8mm, y radius = 3.6mm];
        %\draw[red] (-1.29,1.52) arc [start angle=-200, end angle = 20,
        x radius = 13.75mm, y radius = 3.15mm];
        
        % label of cut of ball surface
        \draw (-1.2,2.2) -- (-0.53,1.83) node at (-1.37,2.37) {$\mathcal{C}_i$};
        
        %shade the spherical cap
        \begin{scope}
            %clip the shading outside the ball
            \clip ({\r*cos(-90)},{\r*sin(-90)}) arc [start angle=-90,end angle=270,radius=\r];
            %draw the disk    
            \fill[red!20] (0,{\r-\H}) circle [x radius={sqrt(\r^2-(\r-\H)^2)}, y radius={0.2*sqrt(\r^2-(\r-\H)^2)}];
            %shade the disk
            \shade[top color=red!70!gray,bottom color=red!10!blue!10!,opacity=0.6]  ({\r},{1.1*\r}) rectangle ++({-2*\r},{-0.1*\r-\H});
        \end{scope}
    
        %draw channel vector
        \draw[densely dashed] (O) -- (0,2);
        %label channel vector
        \draw (1.2,2.2) -- (0,1.72) node at (1.37,2.2) {$\underline{h_i}$};
        \end{scope}
    \end{tikzpicture}
    
    \caption{Surface of hyper-sphere in 2$N$ dimensions. A spherical cap, $\mathcal{C}_i$ is shown (shaded in red). $\mathcal{C}_i$ is formed by projecting the channel vector, $\underline{h_i}$, onto the spherical surface. A colatitude angle of $\theta$ and radius of $\rho^+$ are assumed in forming $\mathcal{C}_i$. Alternatively, $\mathcal{C}_i$ can also be interpreted as the cap formed by intersecting a cone of half-angle $\theta$ with the spherical surface.}
    \label{fig:spherical_cap}
\end{figure}