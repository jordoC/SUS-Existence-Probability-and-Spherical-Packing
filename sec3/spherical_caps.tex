Orthogonality between pairs of vectors can be interpreted geometrically on the surface of a $2N$-dimensional hyper-sphere, given by $\mathcal{S}^{2N}$,  in terms of spherical caps. A spherical cap is formed on the surface of the hyper-sphere of radius $\rho^+$ by by first extending the channel vector $\underline{h_i}$ onto the spherical surface as described in Section \ref{sec:chan_norm}. When $\underline{h_i}$ is extended to the spherical surface, it defines a point $\mathcal{P}_i$ that lives on the surface of the sphere. Then a cone, with its apex at $\mathcal{O}$, with half angle $\theta$, centered along the axis of $\underline{h_i}$ is intersected with the spherical surface. This intersection defines a spherical cap, $\mathcal{C}_i$. This process is illustrated in Fig. \ref{fig:spherical_cap}. 

Spherical caps can be used to describe the the degree of orthogonality between a given vector $\underline{h_i}$ and an arbitrary vector $\underline{h_j}$ by viewing $\mathcal{C}_i$ as a `keep-out zone' $\forall \  \mathcal{P}_j,\ j\neq i \ \in \mathcal{S}^{2N}$. More formally:
\begin{equation}\label{eq:cap_orth}
    \begin{aligned}
    \mathcal{P}_j \not\in \mathcal{C}_i\ \forall j\neq i \in \mathcal{S}^{2N}
    \end{aligned}
\end{equation}
It is important to realize that $\mathcal{C}_i$ is a function of $\theta \in [0,\frac{\pi}{2}]$. That is, $\theta$ is a parameter that describes the degree of orthogonality between vectors: as $\theta \rightarrow \frac{\pi}{2}$ vectors that comply with Eq. (\ref{eq:cap_orth}) are increasingly orthogonal to $\underline{h_i}$. When the condition given in Eq. (\ref{eq:cap_orth}) is satisfied, $\underline{h_i}$ and $\underline{h_j}$ are said to  be $\epsilon$-orthogonal. That is, these two vectors are $\epsilon$-orthogonal to each other to a minimum degree given by the parameter, $\epsilon$.

We can relate the spherical cap interpretation of orthogonality to inner products between vectors on a pair-wise basis. A geometric interpretation is shown in Fig. \ref{fig:cap_triangle}. In this figure, we introduce $\epsilon$, which is related to $\rho^+$, $\theta$:
\begin{equation}\label{eq:epsilon}
    \begin{aligned}
    \epsilon = \rho^+\cos{\theta} \ \ .
    \end{aligned}
\end{equation}
We form the unit vector along $\underline{h_i}$:
\begin{equation}\label{eq:unit_vec}
    \begin{aligned}
    \Hat{\underline{h_i}} = \frac{\underline{h_i}}{\sqrt{\rho^+}} \ \ .
    \end{aligned}
\end{equation}
Thus we can form the inner product between $\Hat{\underline{h_i}}$ and $\Hat{\underline{h_j}}$:
\begin{equation}\label{eq:inner_prod}
    \begin{aligned}
    <\Hat{\underline{h_i}},\Hat{\underline{h_j}}> &= \vert \Hat{\underline{h_j}}^H\Hat{\underline{h_i}} \vert\\
    &\leq \frac{\epsilon}{\rho^+} \ \ .
    \end{aligned}
\end{equation}
The scenario shown in Fig. \ref{fig:cap_triangle} is the case where $\underline{h_j}$ is as close to living entirely in $\mathcal{C}_i$ without doing so. Namely, it shows the case of the equality shown in Eq. (\ref{eq:inner_prod}) where $\mathcal{P}_j$ lives on the circular boundary of $\mathcal{C}_i$ (ie. $\mathcal{P}_j$ may live partially in $\mathcal{C}_i$, but not entirely.). Furthermore, the inequality in Eq. (\ref{eq:inner_prod}) does not hold if $\mathcal{P}_j\in \mathcal{C}_i$.
\begin{figure}
\centering
    \begin{tikzpicture}%[font = \sansmath]
        %define origin, 1-height of cone, radius
        \coordinate (O) at (0,0);
        \def\r{2}
        \def\H{.6}
        
        \begin{scope}
        % ball background color
        \shade[ball color = blue, opacity = 0.5] (0,0) circle [radius = \r];
        
        % cone
        \begin{scope}
            \def\rx{0.71}% horizontal radius of the ellipse
            \def\ry{0.15}% vertical radius of the ellipse
            \def\z{0.725}% distance from center of ellipse to origin
        
            \path [name path = ellipse]    (0,\z) ellipse ({\rx} and {\ry});
            \path [name path = horizontal] (-\rx,\z-\ry*\ry/\z)
                                        -- (\rx,\z-\ry*\ry/\z);
            \path [name intersections = {of = ellipse and horizontal}];
        
            %% radius to base of cone in ball
            %\draw[fill = gray!50, gray!50] (intersectioN) -- (0,0)
            %  -- (intersection-2) -- cycle;
            %% base of cone in ball
            %\draw[fill = gray!30, densely dashed] (0,\z) ellipse ({\rx} and %{\ry});
        \end{scope}
        
        % label of cone
        %\draw (0.25,0.4) -- (0.9,0.1) node at (1.05,0.0) {$q$};
        
        % ball
        \draw (O) circle [radius=2cm];
        
        % label of ball center point
        \filldraw (O) circle (1pt) node[below] {$\mathcal{O}$};
        
        % radius
        \draw[densely dashed] (O) to [edge label = $\rho^+$] (-1.4,1.32);
        \draw[densely dashed] (O) -- (1.4,1.32);
        %label colatitude angle
        \draw[densely dashed] (0.5,0.5) arc [start angle = 45, end angle = 90, x radius = 7mm, y radius = 7mm];
        \draw (2.2, 0.8) -- (0.4, 0.6) node at (2.35,0.8) {$\theta$};
        
        % cut of ball surface
        \draw[red] (-1.35,1.47) arc [start angle = 140, end angle = 40, x radius = 17.6mm, y radius = 14.75mm];
        %\draw[red, densely dashed] (-1.36,1.46) arc [start angle = 170, end angle = 10,
        x radius = 13.8mm, y radius = 3.6mm];
        %\draw[red] (-1.29,1.52) arc [start angle=-200, end angle = 20,
        x radius = 13.75mm, y radius = 3.15mm];
        
        % label of cut of ball surface
        \draw (-1.2,2.2) -- (-0.53,1.83) node at (-1.37,2.37) {$\mathcal{C}_i$};
        
        %shade the spherical cap
        \begin{scope}
            %clip the shading outside the ball
            \clip ({\r*cos(-90)},{\r*sin(-90)}) arc [start angle=-90,end angle=270,radius=\r];
            %draw the disk    
            \fill[red!20] (0,{\r-\H}) circle [x radius={sqrt(\r^2-(\r-\H)^2)}, y radius={0.2*sqrt(\r^2-(\r-\H)^2)}];
            %shade the disk
            \shade[top color=red!70!gray,bottom color=red!10!blue!10!,opacity=0.6]  ({\r},{1.1*\r}) rectangle ++({-2*\r},{-0.1*\r-\H});
        \end{scope}
    
        %draw channel vector
        \draw[densely dashed] (O) -- (0,2);
        %label channel vector
        \draw (1.2,2.2) -- (0,1.72) node at (1.37,2.2) {$\underline{h_i}$};
        % label of ball center point
        \filldraw (0,2) circle (1pt) node[above] {$\mathcal{P}_i$};
        \end{scope}
    \end{tikzpicture}
    
    \caption{Surface of hyper-sphere in 2$N$ dimensions. A spherical cap, $\mathcal{C}_i$ is shown (shaded in red). $\mathcal{C}_i$ is formed by projecting the channel vector, $\underline{h_i}$, onto the spherical surface. A colatitude angle of $\theta$ and radius of $\rho^+$ are assumed in forming $\mathcal{C}_i$. Alternatively, $\mathcal{C}_i$ can also be interpreted as the cap formed by intersecting a cone of half-angle $\theta$ with the spherical surface.}
    \label{fig:spherical_cap}
\end{figure}

\begin{figure}
\centering
    \begin{tikzpicture}%[font = \sansmath]
        %define origin, 1-height of cone, radius
        \coordinate (O) at (0,0);
        \def\r{2}
        \def\H{.6}
        
        \begin{scope}
        % ball background color

        \begin{scope}
            \def\rx{0.71}% horizontal radius of the ellipse
            \def\ry{0.15}% vertical radius of the ellipse
            \def\z{0.725}% distance from center of ellipse to origin
        
            \path [name path = ellipse]    (0,\z) ellipse ({\rx} and {\ry});
            \path [name path = horizontal] (-\rx,\z-\ry*\ry/\z)
                                        -- (\rx,\z-\ry*\ry/\z);
            \path [name intersections = {of = ellipse and horizontal}];
        
            % radius to base of cone in ball
            %\draw[fill = gray!50, gray!50] (intersectioN) -- (0,0) -- (intersection-2) -- cycle;
            % base of cone in ball
            %\draw[fill = gray!30, densely dashed] (0,\z) ellipse ({\rx} and %{\ry});
        \end{scope}
        
        % label of cone
        %\draw (0.25,0.4) -- (0.9,0.1) node at (1.05,0.0) {$q$};

        % label of ball center point
        \filldraw (O) circle (1pt) node[below] {$\mathcal{O}$};
        
        % radius
        \draw[densely dashed] (O) to [edge label = $\rho^+$] (-1.4,1.32);
        %\draw[densely dashed] (O) -- (1.4,1.32);
        %label colatitude angle
        \draw[densely dashed] (-0.5,0.5) arc [start angle = 135, end angle = 90, x radius = 7mm, y radius = 7mm];
        \draw (-2.2, 0.8) -- (-0.4, 0.6) node at (-2.35,0.8) {$\theta$};
        
       
        %\draw[red, densely dashed] (-1.36,1.46) arc [start angle = 170, end angle = 10,
        x radius = 13.8mm, y radius = 3.6mm];
        %\draw[red] (-1.29,1.52) arc [start angle=-200, end angle = 20,
        x radius = 13.75mm, y radius = 3.15mm];
        
        % label of cut of ball surface
        \draw (-1.2,2.2) -- (-0.53,1.83) node at (-1.37,2.37) {$\mathcal{C}_i$};
        
        %shade the spherical cap
        \begin{scope}
            %clip the shading outside the ball
            \clip ({\r*cos(-90)},{\r*sin(-90)}) arc [start angle=-90,end angle=270,radius=\r];
            %draw the disk    
            \fill[red!20] (0,{\r-\H}) circle [x radius={sqrt(\r^2-(\r-\H)^2)}, y radius={0.2*sqrt(\r^2-(\r-\H)^2)}];
            %shade the disk
            \shade[top color=red!70!gray,bottom color=red!10!blue!10!,opacity=0.6]  ({\r},{1.1*\r}) rectangle ++({-2*\r},{-0.1*\r-\H});
        \end{scope}
    
        %draw channel vector
        \draw[densely dashed] (O) -- (0,2);
        
        %epsilon label
        \draw node at (0.2,0.7) {$\epsilon$};
        
        %label channel vector
        \draw (1.2,2.2) -- (0,1.72) node at (1.37,2.2) {$\underline{h_i}$};
         %label channel vector
        \draw (-2.2, 1.3) -- (-0.85,0.85) node at (-2.4,1.3) {$\underline{h_j}$};
        
        % label of ball center point
        \filldraw (0,2) circle (1pt) node[above] {$\mathcal{P}_i$};
        
        % label of point from second vector
        \filldraw (-1.4,1.32) circle (1pt) node[above] {$\mathcal{P}_j$};
        %draw top part of right triangle
        \draw[densely dashed] (-1.4,1.32) -- (0,1.32);
        \end{scope}
    \end{tikzpicture}
    
    \caption{Relationship between spherical cap and inner product of vectors $\underline{h_i},\ \underline{h_j}$}. $\mathcal{P}_j$ lives outside of $\mathcal{C}_i$, however, it is as close as it can be to the cap without living inside it.
    \label{fig:cap_triangle}
\end{figure}