In Section \ref{sec:chan_norm} we described a method of approximately projecting channel vectors onto a spherical surface. In Section \ref{sec:spherical_caps} we used this the event of this projection as a condition for developing a probability that a set of $l$ vectors, $\mathsf{A}$, is $\epsilon$-orthogonal. In this section we set out to extend these results further by considering a larger set of candidate vectors, $\mathsf{C} \ :\ \vert \mathsf{C} \vert = m >l$. This result is particularly useful because it allows us to consider the probability that we can find a group of $l$ vectors that are $\epsilon$-orthogonal as a function of the number of candidates we consider, $n$. One would expect that as more candidates are considered for addition to an $\epsilon$-orthogonal group of size $l$, the probability that such a group meets the constraints and attains minimum size $l$ increases with $n$. However, as more candidates are considered, the complexity associated with forming the $\epsilon$-orthogonal group also increases. 

We can describe the collection of $\epsilon$-orthogonal sets using the criteria developed in the previous sections. Let $\mathsf{S}_\epsilon$ be a collection of $\epsilon$-orthogonal sets. $\mathsf{S}_\epsilon$ is formed by testing all subsets of the set of candidate vectors, $\mathsf{A}\subset\mathsf{C}\ : \ \vert \mathsf{A} \vert = l$, against the channel norm and orthogonality constraints previously developed. Thus, $\mathsf{S}_\epsilon$ is a collection of $l$-length sets. The vectors contained within each of these $l$-length sets are mutually $\epsilon$-orthogonal. In the case that $\mathsf{S}_\epsilon = \lbrace \emptyset \rbrace$, no $\epsilon$-orthogonal sets of cardinality $l$ exist in the set of candidate vectors $\mathsf{C}$. This can be expressed more formally as:
 \begin{equation}\label{eq:S_e}
    \begin{aligned}
        \mathsf{S}_\epsilon = \lbrace \mathsf{A}\ \big|\ | \underline{h_j}^H\underline{h_i} |\ <\ \epsilon \ \text{;} \ \rho^-<\Vert \underline{h_i} \Vert^2 < \rho^+\ \forall \ i \neq j \in \mathsf{A} \rbrace \ \forall \mathsf{A}\subset \mathsf{C}
    \end{aligned}
\end{equation}

The purpose of this discussion is to provide an expression for the probability $Pr[\mathsf{S}_\epsilon \neq \lbrace \emptyset \rbrace]$. We adopt the following notation to describe this probability.
 \begin{equation}\label{eq:p_epsilon}
    \begin{aligned}
        p_\epsilon = Pr[\mathsf{S}_\epsilon \neq \lbrace \emptyset \rbrace]
    \end{aligned}
\end{equation}
More specifically, we want to describe $p_\epsilon$ as a function of $l,\ n,\ \theta$. In order to work up to this result, let us first consider the case that $n=l$. In this case, there is only one non-null subset of $\mathsf{C}$, that is, $\mathsf{C}$ itself. Therefore, the existence probability of $\epsilon$-orthogonal set can be stated in terms of Eqs. (\ref{eq:p_s}), (\ref{eq:p_perp_cdl} or \ref{eq:p_perp_beta}) by recalling that $p_s$ is independent of $p_\perp$:
 \begin{equation}\label{eq:p_epsilon_l}
    \begin{aligned}
        p_{\epsilon,l=n} = p_\perp p_s^l
    \end{aligned}
\end{equation}

However, in the case that $n>l$, we must now consider more than one subset of $\mathsf{C}$. In order to do so, we define the the indicator random variable that returns a binary value indicating whether or not $\mathsf{A}\in \mathsf{S}_\epsilon$:
 \begin{equation}\label{eq:orth_indicator}
    \begin{aligned}
        \textbf{1}_{\mathsf{A}} = 
        \begin{cases}
            1,& \text{if } \mathsf{A} \in \mathsf{S}_\epsilon\\
            0,              & \text{otherwise}
        \end{cases}
    \end{aligned}
\end{equation}

The number of $\epsilon$-orthogonal sets in the collection $\mathsf{S}_\epsilon$ may be expressed in terms of a sum of these indicator variables:
 \begin{equation}\label{eq:indicator_sum}
    \begin{aligned}
        K_\epsilon^{(l)} = \sum_{\mathsf{A}\subset\mathsf{C}}1_{\mathsf{A}} \ : \ \vert \mathsf{A} \vert = l
    \end{aligned}
\end{equation}
One can also observe that the expression given in Eq. ($\ref{eq:p_epsilon}$) is equivalent to $Pr[ K_\epsilon^{(l)} \neq 0]$.

It is very important to note at that the argument from one iteration of the sum to the next in Eq. (\ref{eq:indicator_sum}) are not necessarily independent. This follows from the fact that the arbitrary subsets of $\mathsf{C}$ may have intersections. Therefore, the random variables that correspond to these sets will also be dependent. For example, consider tow subsets of $\mathsf{C}$, $\mathsf{A}\subset\mathsf{C}$ and $\mathsf{B}\subset\mathsf{C}$. In the case that $\mathsf{A}\cap\mathsf{B} = \lbrace \emptyset \rbrace$, the the indicator random variables $1_{\mathsf{A}}$ and $1_\mathsf{B}$ will also be independent. However, when $\mathsf{A}\cap\mathsf{B} \neq \lbrace \emptyset \rbrace$, then the random variables $1_{\mathsf{A}}$ and $1_\mathsf{B}$ are mutually dependent. It is important to account for this dependence for the sake of developing an accurate bound on $p_\epsilon$. An approach based on random graphing and dependence graphing, similar to \cite{Swannack2005}, \cite{Janson2004} will be taken to overcome this problem.